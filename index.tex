\documentclass[11pt,dvipdfmx,b5paper,oneside,report,uplatex]{jsbook}


\usepackage{color}
\usepackage{here}
\usepackage{framed}
\usepackage{tcolorbox}
\usepackage{quotchap}
\usepackage{pdfpages}
\usepackage[hidelinks]{hyperref}
\usepackage{pxjahyper}
\usepackage{titlesec}
\usepackage{picture}
\usepackage{tikz}
\usepackage{graphicx}
\usepackage{geometry}
\usepackage{url}
\usepackage{pdfpages}


\tcbuselibrary{breakable,listings}
\definecolor{shadecolor}{gray}{0.80}

% 余白を狭くする
\geometry{left=20mm,right=20mm,top=25mm,bottom=25mm}

% section
\titleformat{\section}[block]{}{}{0pt}
{
  \definecolor{teal}{gray}{0.30}
  \begin{picture}(0,0)
    \put(-10,-5){
      \begin{tikzpicture}
        \fill[teal] (0pt,0pt) rectangle (5pt,19pt);
      \end{tikzpicture}
    }
    \put(-10,-5){
      \color{teal}
      \line(1,0){\hsize}
    }
  \end{picture}
  \hspace{0pt}
  \sf \Large \thesection
  \hspace{0pt}
}

% 図表見出し
\renewcommand{\tablename}{\textcolor{gray}{▼} 表}
\renewcommand{\figurename}{\textcolor{gray}{▲} 図}

\begin{document}

\include{./content/01-title}
%目次を自動的に作る。
\tableofcontents
\chapter{初めに}

さて、みなさん。Webサイトは作られていますか?
Webサイトを作る時にそのままHTMLを触ってGithubなどで管理をしてもいいのですが、やはりGithubを使ったことない人にとってはWebサイトの更新だけでかなり大変な作業になってしまいます。

そこで、今サークルの情報共有で用いているesaを用いてWebサイトを更新したら簡単に、誰でも更新できるのではないかと考え、実装してみることにしました。
そこまで難しくないので、ぜひ参考にしてみてください。
\chapter{実際の動作しているサイト}

https://github.com/SystemEngineeringTeam/BlogSiteMarkDown

https://esa.harutiro.net/
\chapter{用語まとめ}

\section{esa}

  \begin{figure}[H]
    \centering
    \includegraphics[width=6cm]{./image/02-chap3/esa.png}
    \caption{esaのホームページの写真}
    \label{chap3-esa-image}
  \end{figure}


  \begin{tcolorbox}[title=esaとは]
    esaとは、2014年に設立された合同会社esaの「情報共有ツール」です。
    esaは「不完全でも早い段階でチームに共有し、更新を重ねることでより良い情報に育つ」という発想のもと生まれました。そのため「Share(公開)」「Develop(更新して情報を育てる)」「Organize(育った情報を整理)」の3つの流れで設計されています。
    現在は3,000社を超える企業に導入されており、主に情報の蓄積やWIP機能(書いている途中でも共有する機能)を用いて、業務の効率化を実現している企業が多いです。
    \cite{esaとは} \cite{公式esaWeb}

  \end{tcolorbox}

\section{hugo}

  \begin{figure}[H]
    \centering
    \includegraphics[width=6cm]{./image/02-chap3/hugo.png}
    \caption{hugoのホームページの写真}
    \label{chap3-hugo-image}
  \end{figure}

  \begin{tcolorbox}[title=hugoとは]
    HugoはGo言語で実装された「Webサイト構築フレームワーク」で、最初の公開は2013年という比較的新しいツールだ。コンテンツ管理システムではなく「Webサイト構築フレームワーク」と名乗っているとおり、コンテンツの管理ではなく、Webサイトで使われるHTMLファイルやRSSファイルなどの生成に特化した機能を備えている。
    \cite{hugoとは} \cite{hugo公式}
  \end{tcolorbox}

\section{GitHub Actions}

  \begin{figure}[H]
    \centering
    \includegraphics[width=6cm]{./image/02-chap3/githubActions.png}
    \caption{GitHub Actionsのホームページの写真}
    \label{chap3-githubAction-image}
  \end{figure}

  \begin{tcolorbox}[title=GitHub Pagesとは]
    GitHub Actionsで、ソフトウェア開発ワークフローをリポジトリの中で自動化し、カスタマイズし、実行しましょう。 CI/CDを含む好きなジョブを実行してくれるアクションを、見つけたり、作成したり、共有したり、完全にカスタマイズされたワークフロー中でアクションを組み合わせたりできます。
    \cite{githubAction}
  \end{tcolorbox}

\section{GitHub Pages}

  \begin{tcolorbox}[title=hugoとは]
    GitHub Pages は、GitHub のリポジトリから HTML、CSS、および JavaScript ファイル を直接取得し、任意でビルドプロセスを通じてファイルを実行し、ウェブサイトを公開できる静的なサイトホスティングサービスです。
    \cite{githubPages}
  \end{tcolorbox}


\section{CMS}

  \begin{tcolorbox}[title=CMSとは]
    「CMS」とは、「Contents Management System:コンテンツ・マネジメント・システム」の略で、簡単にいうとWebサイトのコンテンツを構成するテキストや画像、デザイン・レイアウト情報(テンプレート)などを一元的に保存・管理するシステムのことです。
    \cite{cmsとは}
  \end{tcolorbox}


\chapter{技術構成}
基本的には、esaによりマークダウンを作成して、それをGithubに保存して、GithubActionsを用いてhugoに出力して、Webサイトに公開する方法です。

ほぼほぼノーコードでできる構成になっているので、そこまで手間がかからずに作成をすることができます。

\begin{figure}[H]
  \centering
  \includegraphics[width=14cm]{./image/02-chap4/flow.png}
  \caption{実際に用いた技術構成の図}
  \label{chap4-flow-image}
\end{figure}

\include{./content/02-test}






\end{document}