\chapter{おわりに}

今回は、esaをCMSとして使い、GithubActionsで静的なWebサイトを作成して、GithubPagesで公開するということを行いました。

esaでは、設定からAPIを使うことで簡単にGithubに記事を投稿することができます。
GithubActionsでは、Githubにpushされたら自動でビルドを行い、GithubPagesに公開するということを行いました。
hugoを使うことで、テンプレートを使って簡単に綺麗なWebサイトを作成することができます。

esa自体は、とてもいいサービスで、さまざまなAPIが公開されており、Slackと連携をしたり、Githubと連携をしてバックアップを簡単にとることもできます。
複数人で一斉に編集をすることもできるため、私たちのサークルでは、情報共有のために使っています。

あまりesaをCMS化しようという試みがないかもしれませんが、大変便利なものなのでぜひ使ってみてはいかがでしょうか?
2022年/12月時点では、esaは学生団体でしたら、1年間無料で使えて更新も無料らしいので、ぜひ使ってみてください。

