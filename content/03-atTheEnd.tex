% あとがき用のコマンド集
\renewcommand{\textbf}[1]{{\bfseries\sffamily#1}}
\newcommand{\bhline}[1]{\noalign{\hrule height #1}}  

\newpage
\thispagestyle{empty}
\section*{奥付け}

今回はこの本を手に取っていただきありがとうございました。

自分は普段はAndroidを中心としたモバイル開発を行っているので、本当はAndroidに関する記事を書きたかったのですが、時間の関係で過去にアドベントカレンダーに書いた記事を使うことになりました。
本当に残念無念。

ですが、この一年サークルで様々なことを学び、自分の知識が増えたと思います。
その一つがサーバー側の知識だと思います。

シス研ではOBの方がサーバーを管理してくださっているのですが、ここ三年ほどは知識の引き継ぎがされておらず、この一年でサルベージ作業をしていました。
その中で、過去使われていたサーバーの知識であったり、現代のモダンな構築方法を学びました。
最近はクラウド化されているので、本当に様々なサービスを簡単に触れるようになりました。

今後も、サーバーの知識や、モバイル開発の知識など様々な分野を深めていきたいと思います。

\begin{table}[b]%
	\centering%
	\begin{tabular}{lcll}%
		\multicolumn{4}{c}{ {\LARGE \shortstack{esaをCMSにしてGitHub ActionsとHugoを\\用いて自動でホームページを更新する方法の考案}} }\\
		\bhline{1pt}
		発行日 && 2023年 5月 28日 & (初版)	\\
%		 && 2019年 $\phantom{1}$2月 28日 &	(第二版)\\
		サークル && 愛知工業大学 システム工学研究会 &	\\
    Instagram ID && @ait.sysken& 	\\
		Twitter ID && @set$\_$official &	\\
		QiitaOrganizationURL && https://qiita.com/organizations/sysken &	\\
		代表 && 牧野遥斗 & \\
		代表者メールアドレス && harutiro2027@icloud.com & \\
		著者 && 牧野遥斗 (Twitter: @minesu1224)  &	\\
		印刷所 && しまや出版 & \\
		\bhline{1pt}
		\multicolumn{4}{c}{ {※本書の無断複写、複製、データ配信はかたくお断りいたします。} }	
	\end{tabular}%
\end{table}%